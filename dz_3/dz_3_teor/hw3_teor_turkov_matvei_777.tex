% commentary

\documentclass[a4paper, 11pt]{article} %type 

% Russian Language

\usepackage[T2A]{fontenc}
\usepackage[utf8]{inputenc}
\usepackage[english, russian]{babel}

% Math

\usepackage{amsmath, amsfonts, amssymb, amsthm, mathtools} 
\usepackage{wasysym} 
\usepackage{tikz}
\usetikzlibrary{automata,positioning}
\usepackage{hyperref} 
\usepackage{algorithm}
\usepackage{algorithmic}
% Headline

\author{Турков Матвей}
\title{Домашнее задание №3}
\date{\today}

\usepackage{tikz}
\newcommand*\circled[1]{\tikz[baseline=(char.base)]{
            \node[shape=circle,draw,inner sep=2pt] (char) {#1};}}
            
            
            \begin{document}

\begin{center}
  \begin{Large}
    Домашнее задание 3
  \end{Large}
\end{center}
\begin{center}
  \begin{Large}
    Турков Матвей, группа 777
  \end{Large}
\end{center}

	\hfill \break	
	
\subsection*{\circled{1}} 
\subsubsection*{\textit{Решение:}}
 \begin{enumerate}
   \item Сошлюсь на существующее доказательство , которое я понял и не вижу смысла просто перепечатывать . 
   \href{https://tinyurl.com/y62xsn6x}{Тык}
   
   \item $\mathcal{P} \in co-\mathcal{NP} \rightarrow L \in \mathcal{P} \Rightarrow L \in co-\mathcal{NP}$. Ясно, что $\mathcal{P} \subseteq \mathcal{NP}$, так как $\forall L \in \mathcal{P}\, \exists$ полиномиальный алгоритм, разрешающий язык(вроде как было показано на семинаре). Рассмотрим язык $co-\mathcal{NP} = \{L | \overline{L} \in \mathcal{NP}\}$.В силу замкнутости языков из  $ \mathcal{P}$ относительно дополнения и $\mathcal{P} \subseteq \mathcal{NP} \rightarrow \mathcal{P} \subseteq co-\mathcal{NP}$
   
 
 \end{enumerate}

\subsection*{\circled{2}} 
\subsubsection*{\textit{Решение:}}
\begin{enumerate}

 \item $\mathrm{3-SAT}$ .Данный язык лежит в классе $\mathcal{NP}$,$\forall x\exists $ сертификат $y$, при котором $x$ истина, если такого нет, то $y$ ,например, набор нулей. 
 
 \item $\mathrm{VCOVER}$ - Сертификат $y$ в данном случае - то вершинное покрытие мощности $k$, или любое множество, например, пустое, если не существует  Легко понять, что МТ работает за полиномиальное время и при этом $|y| = O(|(G,k)| ^ {c})$ 

 \item Язык, состоящий из графов, содержащий эйлеров путь.  Сертификат $y$ набор ребер, образующий эйлеров путь, а МТ $A(x, y)$ будет идти по графу и убирать встретившиеся ребра, если встретилось дважды или не было пройдено - МТ вернет $0$, иначе $1$, в случае графов, в которых нет эйлерова пути сертификатом произвольное множество ребер
Данный алгоритм полиномиален

 \item Язык, состоящий из описаний всех ориентированных взвешенных графов, в которых нет цикла отрицательной длины. Сертификат $y$  вершины, образующие цикл отрицательной длины, если такого не существует, то сертификат произвольное множество. МТ проверяет содержит ли граф ребра из $y$, образют ли они цикл отрицательной длины, идя последовательно и считая сумму ребер. Если сумма $< 0$, тогда МТ вернет $1$ иначе $0$. Данный алгоритм полиномиален и $|y| = O(|x|^{c})$
 
 \item Язык состоящий из пар $(G, \omega)$, где $G$ - набор правил, описывающих КС-граматику над алфавитом $\{1, 2\}$, а $\omega \in \{1,2\}^*$ - слово невыводимое в этой граматике. По КС-грамматике легко строится МП автомат за полином. Если слово лежит в языке, то запустим MП автомат $G$ на $\omega$,в качестве сертификата $y$  переходы, по которым получим $\omega$, если оно выводимо в граматике или любое множество, если нет. МП автомат остановится в принимающем состоянии МТ вернет $0$ и $1$ соответственно  . Сертификат $|y| = O(|G|^{c})$, язык лежит в $\mathcal{NP}$
 
 \item $\mathrm{PLANARITY}$ - язык описаний планарных графов. Данный язык принадлежит классу $\mathcal{P}$, так как достаточно посчитать $V - E + F$ Таким образом, язык принадлежит $\mathcal{P}$ и $\in \mathcal{NP}$
\end{enumerate}

\subsection*{\circled{3}} 
\subsubsection*{\textit{Решение:}}
\begin{enumerate}

 \item $\mathrm{TAUT}$. Докажем, что язык $L = \overline{\mathrm{TAUT}}$ лежит в $\mathcal{NP}$ . Рассмотрим МТ $A(x, y)$, выводяющую значение $\overline{x(y)}$.Данная МТ работает за полином. $\forall$ формулы $\, \exists$  сертификат: для общезначимых - любой,  для остальных- набор, на котором формула обращается в ноль, при этом также$\exists$ МТ $A(x, y)$, выдающая $1$ , когда $x$ - не является общезначимой за полиномиальное время и $|y| = O(|x|^{c})$. Принадлежность доказана
 
 \item $L$ - язык, состоящий из пар $(G, m)$, где $G$ - описание графа такого, что для любых $m$ вершин найдется ребро, соединяющее хотя бы 2 из них. Рассмотрим сертификат$y$ для $\overline{L}$ . Это набор из $m$ вершин, причем МТ $A((G, m), y)$ для каждой пары вершин из $y$ проверяет лежит ли эта пара в множестве ребер графа $G$. Эта операция полином по входу и $|y| = O(|(G, m)|^{c})$. Принадлежность доказана
 
 \item $\mathrm{FACTORING}$ - язык натуральных троек $(a, b,c)$ таких, что $a$ имеет простой делитель из $[b,c]$. Рассмотрим МТ $A(x,y)$ для $L = \overline{\mathrm{FACTORING}}$, в которой $x = (a,b,c)$ которая используя решето Эратосфена проверяет есть ли  $[b,c]$ простые числа, если
нет, то $1$. Проверим делимость числа $a$ на $y$ простым перебором $z \in [1,a]$ , если нет такого $z$ , то возвращаем $1$, иначе $0$ . Таким образом, данная МТ работает за полином. Сертификатом простое число из $[b,c]$, если оно существует и, $0$, если нет. $|y| =   O(|x|^{c})$ Принадлежность доказана

 \item Язык описаний графов в в которых есть клика на $2019$ элементов. Алгоритм из $\mathcal{P}$, так как существует ровно  $C^{2019}_n = O(n^{2019})$ способов выбрать $n$ из 2019, при этом каждая клика  проверяется за полином на "кликовость" , $\rightarrow \mathcal{P}$, как было доказано в первой задаче и  $ \in co-\mathcal{NP}$

\end{enumerate}
\subsection*{\circled{5}} 
\subsubsection*{\textit{Решение:}}
\par Покажем полиномиальность 
\par Получим необходимое из порождающего элемента за  $O(\log{p})$ умножений , $O(\log{p})$ получений остатка по необходимому модулю,  
поэтому $O(\log^3{p})$(текущая оценка сложности). Всего чисел $k^{\frac{p-1}{p_i}}$ $O(\log{p})$, значит проверка выполняется за $O(\log^4{p})$ ($p_i$ простые делители $p-1$). Проверка того, что простыми делителями $p-1$ являются $p_i$ делается за $O(\log^3{p})$. Отсэда получим необходимую рекуренту :

\[T(p) = O(\log{p}) + \sum_{i=1}^k T(p_i)\]

Из свойств неравенств , получим ,что сумму на каждом уровне можно оценить как $O(\log^4{p})$, всего таких уровней $\log{p}$. А значит, данная задача имеет сложность  $O(\log^5{p})$ и является полиномиальной .

\[100091236 = 2^2 \cdot 7 \cdot 3574687 = 2^2 \cdot 7 \cdot 2 \cdot 3 \cdot 233 \cdot 2557 + 1 \cdot 2^2 \cdot 7 \]
\[232 = 2^3 \cdot 29 =2^3 \cdot 2^2 \cdot 7 + 1 \cdot 2^3 \]
\[2556 = 2^2 \cdot 3^2 \cdot 71 = 2^2 \cdot 3^2 \cdot 2 \cdot 5 \cdot 7 + 1 \cdot 2^2 \cdot 3^2 \]

Порождающие 
\begin{align*}
&100091237 -\, 2\\
&3574687 - \,62\\
&2557 -\, 2\\
&233 -\, 2\\
&71 -\, 14\\
&29 -\, 2\\
&7 -\, 2\\
\end{align*} 

\subsection*{\circled{6}} 
\subsubsection*{\textit{Решение:}}
\par Предоставляю две ссылки , которые содержат решение данной задачи.   \par\href{https://bit.ly/2SBvEmS}{Раз} (Th 2) и  \href{https://bit.ly/2Xwzy45}{Два}. \par Я вроде как в них разобрался, поэтому посчитал нужным не заниматься простым переписыванием 

\end{document}