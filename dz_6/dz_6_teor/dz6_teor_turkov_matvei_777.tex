% commentary

\documentclass[a4paper, 11pt]{article} %type 

% Russian Language

\usepackage[T2A]{fontenc}
\usepackage[utf8]{inputenc}
\usepackage[english, russian]{babel}

% Math

\usepackage{amsmath, amsfonts, amssymb, amsthm, mathtools} 
\usepackage{wasysym} 
\usepackage{tikz}
\usetikzlibrary{automata,positioning}
\usepackage{hyperref} 
\usepackage{algorithm}
\usepackage{algorithmic}
\usepackage[normalem]{ulem}
% Headline

\author{Турков Матвей}
\title{Домашнее задание №6}
\date{\today}

\usepackage{tikz}
\newcommand*\circled[1]{\tikz[baseline=(char.base)]{
            \node[shape=circle,draw,inner sep=2pt] (char) {#1};}}
            
            
            \begin{document}

\begin{center}
  \begin{Large}
    Домашнее задание 6
    \end{Large}
\end{center}
\begin{center}
  \begin{Large}
    Турков Матвей, группа 777
  \end{Large}
\end{center}

	\hfill \break	

На удивление, в этом домашнем задании уж слишком много пересечений с семинаром Горбунова по данной теме, поэтому я решил не техать, уже затеханное другим семинаристом (press F to pay \sout{dis}respect)

\subsection*{\circled{1}} 
\subsubsection*{\textit{Решение:}}
\begin{enumerate}
\item Семинар 08, стр 4-5
\item Пользуясь свойствами матриц и прошлым пунктом, получим
\begin{align*}
a_i =\frac{1}{n} \sum_{k = 0}^{n-1}y_k w_n^{-ik}
\end{align*}
\end{enumerate}

\subsection*{\circled{2}} 
\subsubsection*{\textit{Решение:}}
\begin{enumerate}
\item Перемножьте многочлены $1+2x+3x^2, 3x-4+6x^3$ с помощью ДПФ $(w_8 = e^{\frac{2\pi i}{8}})$
\newline Для первого многочлена распишем подробно, остальные , аналогично 
\[
DFT
\begin{pmatrix}
1 \\ 0
\end{pmatrix} = 
\begin{pmatrix}
1 \\ 1
\end{pmatrix};
DFT
\begin{pmatrix}
3 \\ 0
\end{pmatrix} = 
\begin{pmatrix}
3 \\ 3
\end{pmatrix};
DFT
\begin{pmatrix}
2 \\ 0
\end{pmatrix} = 
\begin{pmatrix}
2 \\ 2
\end{pmatrix};
DFT
\begin{pmatrix}
0 \\ 0
\end{pmatrix} = 
\begin{pmatrix}
0 \\ 0
\end{pmatrix}
\]
\[DFT
\begin{pmatrix}
1 \\ 3 \\0 \\0
\end{pmatrix} = 
\begin{pmatrix}
4 \\ 1 + 3e^{\frac{2i\pi}{4}}\\ -2 \\ 1 - 3e^{\frac{2i\pi}{4}}
\end{pmatrix};
DFT
\begin{pmatrix}
2 \\ 0 \\0 \\0
\end{pmatrix} = 
\begin{pmatrix}
2 \\ 2 \\ 2 \\ 2
\end{pmatrix};
\]

\[DFT
\begin{pmatrix}
1 \\ 2 \\ 3 \\ 0 \\ 0 \\ 0 \\ 0 \\ 0
\end{pmatrix} = 
\begin{pmatrix}
6\\ 1 + 3e^{\frac{2i\pi}{4}} + 2e^{\frac{2i\pi}{8}} \\ -2 + 2e^{\frac{4i\pi}{8}}  \\ 1 - 3e^{\frac{2i\pi}{4}} +2e^{\frac{6i\pi}{8}}   \\ 2 \\ 1 + 3e^{\frac{2i\pi}{4}} -  2e^{\frac{2i\pi}{8}}  \\ -2  - 2e^{\frac{4i\pi}{8}} \\ 1 - 3e^{\frac{2i\pi}{4}}  - 2e^{\frac{6i\pi}{8}}
\end{pmatrix}
\]

Аналогично, для второго 

\[DFT
\begin{pmatrix}
-4 \\ 3 \\ 0 \\ 6 \\ 0 \\ 0 \\ 0 \\ 0
\end{pmatrix} = 
\begin{pmatrix}
5\\ -4 -\frac{3}{\sqrt{2}} + \frac{9}{\sqrt{2}}i\\ -4 -3i  \\-4+\frac{3}{\sqrt{2}} + \frac{9}{\sqrt{2}}i  \\ -13 \\-4+\frac{3}{\sqrt{2}} - \frac{9}{\sqrt{2}}i   \\ -4  + 3i \\ -4 -\frac{3}{\sqrt{2}} - \frac{9}{\sqrt{2}}i
\end{pmatrix}
\]

\[
DFT(1+2x+3x^2) \times DFT(3x-4+6x^3) = 
\begin{pmatrix}
30\\-16-19\sqrt{2} -(6+4\sqrt{2})i \\ 14 -2i\\ -16+19\sqrt{2} +(6-4\sqrt{2})i\\-26 \\-16+19\sqrt{2} +(-6+4\sqrt{2})i\\14 + 2i\\-16-19\sqrt{2} +(6+4\sqrt{2})i
\end{pmatrix}
\]

\[
invDFT(DFT(1+2x+3x^2) \times DFT(3x-4+6x^3)) = 
\begin{pmatrix}
-4 \\ -5 \\-6\\15 \\12 \\18 \\ 0 \\ 0
\end{pmatrix}
\]


\end{enumerate}

\subsection*{\circled{3}} 
\subsubsection*{\textit{Решение:}}
\begin{enumerate}
\item Семинар 08, стр 6-7 
\item Семинар 08, стр 6-7 
\end{enumerate}

\subsection*{\circled{4}} 
\subsubsection*{\textit{Решение:}}
\begin{enumerate}
\item $DFT(c_0, c_1, ... c_{n-1}) \times DFT(x_0, x_1, ... x_{n-1}) = DFT(b_0, b_1, ... b_{n-1})$
\[DFT(c_0, c_1, ... c_{n-1}) \times DFT(x_0, x_1, ... x_{n-1}) = \]\[
\begin{pmatrix}
\sum c_i \cdot \sum x_i \\ \sum c_iw_n^{i} \cdot \sum x_i w_n^{i} \\ ... \\ \sum
 c_iw_n^{i(n-1)} \cdot \sum x_i w_n^{i(n-1)}
\end{pmatrix} = DFT(b_0, b_1, ... b_{n-1})
\]
\item Семинар 08, стр 8
\end{enumerate}



\end{document}