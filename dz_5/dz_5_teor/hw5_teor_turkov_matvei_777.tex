% commentary

\documentclass[a4paper, 11pt]{article} %type 

% Russian Language

\usepackage[T2A]{fontenc}
\usepackage[utf8]{inputenc}
\usepackage[english, russian]{babel}

% Math

\usepackage{amsmath, amsfonts, amssymb, amsthm, mathtools} 
\usepackage{wasysym} 
\usepackage{tikz}
\usetikzlibrary{automata,positioning}
\usepackage{hyperref} 
\usepackage{algorithm}
\usepackage{algorithmic}
% Headline

\author{Турков Матвей}
\title{Домашнее задание №5}
\date{\today}

\usepackage{tikz}
\newcommand*\circled[1]{\tikz[baseline=(char.base)]{
            \node[shape=circle,draw,inner sep=2pt] (char) {#1};}}
            
            
            \begin{document}

\begin{center}
  \begin{Large}
    Домашнее задание 5
    \end{Large}
\end{center}
\begin{center}
  \begin{Large}
    Турков Матвей, группа 777
  \end{Large}
\end{center}

	\hfill \break	

\subsection*{\circled{1}} 
\subsubsection*{\textit{Решение:}}
\subparagraph{1.}
$a = 12^{14^{18^3}} \mod 19 = 12^{14^{18}} \cdot 12^{14^{18}} \cdot 12^{14^{18}} \mod 19 $
\[14^{18^{3}} = 14^{6^{3}\cdot 3^{3}} = 2^{6k}\cdot7^{6k} \mod 18 = 64^k \mod 18 = 10\mod 18\]
\[12^{14^{18^{3}}} = 12^{10+18m} = 12^{18m}\cdot 12^{10} \mod 19 = 12^{10} \mod 19 = (12 \cdot 12^4)^2 \mod 19 = 7  \]

\subparagraph{2.}
\[14^{11^{289}} \mod 25\]
\[289 mod \phi(25) = 9\]
\[14^{11^{9}} \mod 25\]
\[11^9 \mod \phi(25) \]
\[9 = 1\mod \phi(20)\]
\[11^9 = 11 \mod 20 = 11\]
\[14^{11^{9}} = 14^{11} \mod 25\]
\[((14^2\mod 25)\cdot (14^2 \mod 25) \cdot (14^2 \mod 25) \cdot (14^2 \mod 25) \cdot (14^2 \mod 25) \cdot 14)\mod 25\]
\[((21^2\mod 25)\cdot(21^2\mod 25)\cdot 21 \cdot 14) \mod 25\]
\[((16^2\mod 25)\cdot 21 \cdot 14) \mod 25\]
\[(6\cdot 21 \cdot 14) \mod 25\]
\[(9\cdot 21 ) \mod 25\]
\[14^{11^{289}} = 14 \mod 25\]
\subparagraph{3.}
\[7^2 = 1 \mod 24) \rightarrow 7^{14^{20^9}} = 1 \mod 24 \]

\subsection*{\circled{3}} 
\subsubsection*{\textit{Решение:}}
\[\Sigma_{1}^{m}(i) = \frac{1 + m}{2} \cdot m \mod m?\]
Четный случай
\[m = 2k;\, \frac{1 + 2k}{2} \cdot 2k \mod 2k = k + 2k^2 \mod 2k = k  \mod 2k = m/2\mod m\]
А так же нечетный случай 
\[ m = 2k+1;\, \frac{1 + 2k + 1}{2} \cdot (2k+1) \mod 2k+1 = (k+1)(2k+1) \mod 2k +1 \]
\[= (k+1) \mod 2k+1 = (m-1)/2 \mod m\]
\subsection*{\circled{5}} 
\subsubsection*{\textit{Решение:}}
\begin{equation}
\begin{cases}
x \mod 36 = 24\\
x \mod 54 = 45 \\
x \mod 107 = 53
\end{cases}
\end{equation} 
\begin{equation}
\begin{cases}
x = 36k + 24\\
x = 54l + 45 \\
x = 107m + 53
\end{cases}
\end{equation}
Подставим x из второго в первое:
\[54l + 45 = 36k + 24 \]
Слева нечетное, справа четное => решений нет
\subsection*{\circled{6}} 
\subsubsection*{\textit{Решение:}}
\par Поскольку $m$ и $n$ не взаимопросты и , по определению алгоритма , $n=pq$, то $m$ делимо либо на $p$, либо  на $q$. Пусть для определенности, $m=ps$, где $s$ некое число, причем необязательно простое. 
\par Рассмотрим процесс шифрования
\begin{align*}
&c = E(m) = m^e \mod n = (ps)^e \mod (pq)\\
&p^eq^e\mod(pq) = p^e\mod(pq)\cdot s^e\mod(pq)\\
\end{align*} 
Теперь рассмотрим $p^e\mod(pq)$. Ясно, что 
\begin{align*}
&\forall e : p^e = 0\mod(pq)
\end{align*}
А значит, $E(m)$ всегда будет возвращать 0 и мы не сможем ничего зашифровать.


\end{document}