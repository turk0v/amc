\documentclass[a4paper, 12pt]{article}
\usepackage[utf8]{inputenc}
\usepackage[T2A]{fontenc}
\usepackage[english, russian]{babel}
\usepackage{amsmath, amsfonts, amssymb, amsthm, mathtools, forest}
\usepackage{graphicx}
\usepackage{wrapfig}
\usepackage{multirow}
\usepackage{forest}
\usepackage{float}
\usepackage{hyperref} 



\usepackage{tikz}
\newcommand*\circled[1]{\tikz[baseline=(char.base)]{
            \node[shape=circle,draw,inner sep=2pt] (char) {#1};}}

\begin{document}

\begin{center}
  \begin{Large}
    Домашнее задание 4
  \end{Large}
\end{center}
\begin{center}
  \begin{Large}
    Турков Матвей, группа 777
  \end{Large}
\end{center}

\subsection*{\circled{1}} 
\subsubsection*{\textit{Решение:}}

\begin{enumerate}
\item Так как разница между ДМТ и НДМТ лишь в функции переходов $\delta$, то мы можем повторить алгоритм доказательства замнкнутости $P$, но уже на НДМТ, что, по определению, даст замкнутость $NP$-языков по итерации.   


\item Для любого языка $L\in P$ верно, что $\Bar{L}\in P \Rightarrow \Bar{L} \in NP\Rightarrow L \in co-NP$.
\end{enumerate}
\subsection*{\circled{2}} 
\subsubsection*{\textit{Решение:}}

\begin{enumerate}

\item Нашим сертификатом будет набор значений, на которых формула выполнима. Верификатор - функция, которая будет выполнять следующий алгоритм: идем до первой закрывающей скобки, после нее возвращаемся назад, подставляя в пропозицонные символы значения и, встретив открывающую скобку, вычисляем значение скобки и заменяем ее на нужную константу. Каждый раз, сворачивая скобку, мы уменьшаем формулу минимум на 1 символ/связку. Всего проходов может быть $O(n^2)$. Следовательно, данная функция полиномиальна.


\item В виде сертификата предоставим вектор вершин размера $|V|$(1 - вершина входит, 0 - не входит). Далее проходом по графу, просматривая все ребра, проверяем данный сертификат.



\item Сертификат - посл-ть вершин эйлерова пути в данном графе. Проходим один раз по графу, проверяя условие эйлерова пути и получаем ответ.


\item Применяя алгоритм Форда-Беллмана мы за полином удостоверимся в наличии или отсутствии цикла с отрицательными весами ($\in P$). Если после $n-1$ фазы мы выполним ещё одну фазу, и на ней произойдёт хотя бы одна релаксация, то граф содержит цикл отрицательного веса, достижимый из $v$; в противном случае, такого цикла нет.


\item Любую КС-грамматику можно за полином свести к НФХ(см. 2-ое ДЗ). Далее, используя алгоритм Кока-Янгера-Касами узнаем выводимость слова ($\in P$).


\item По теореме Понтрягина-Куратовского граф планарен т.к. он не содержит подграфов, гомеоморфных полному графу из пяти вершин $K_5$ или графу «домики и колодцы» $K_{3,3}$. Т.к. данная теорема - критерий, то $L_{planarity} \in NP \bigcap co-NP$. Следовательно, доказав принадлежность какому-то из классов, сразу докажем и принадлежность другому.

Сертификат некомпланарности - набор вершин подграфа Куратовского. Его же вычисление может быть сделано методом добавления пути/вершин/ребер (см. Википедия). Все эти алгоритмы работают за полином, следовательно, $L_{planarity} \in P$.

\end{enumerate}

\subsection*{\circled{3}} 
\subsubsection*{\textit{Решение:}}

\[L \in co-NP \Rightarrow \Bar{L} \in NP\]
\begin{enumerate}

\item $\Bar{L}$ - нетавтологичные формулы. Следовательно, сертификат - вектор значений, на котором формула ложна. Далее, аналогично задаче $3-SAT$, получаем ответ за полином.

\item $\Bar{L}$ - описания графов, таких, что $\exists k: \forall (v_i, v_j) \in E $???Ненорм условие описано

Сертификат - вектор ребер с $k$ единицами. Проходя по графу один раз мы проверяем требуемое условие за полином.

\item Сертификат - массив простых делителей $p_i$ числа $a$. Проверяем принадлежность хотя бы одного из них отрезку $[b, c]$.

\item Необходимым и достаточным условием для существования клики размера $k$ является наличие независимого множества размера не менее $k$ в дополнении графа. Сертификат - вектор вершин дополнения графа, входящих в независимое подмн-во размера не меньше $k$. Далее проверяем условие клики.

\item Смотри $2(6)$.

\end{enumerate}
\subsection*{\circled{4}} 
\subsubsection*{\textit{Решение:}}

\[L \in co-NP \Rightarrow \Bar{L} \in NP\]

$L$ - язык общезначимых предикатов, тогда $\Bar{L}$ - необщезначимые предикаты $\to$ $\exists x: g(x) = 0$.

Предоставляя в виде сертификата вектор значений пропозициональных символов мы, аналогично задаче $2(i)$ проходим по предикату, сводя его к константе. Следовательно, $L \in co-NP$.

\subsection*{\circled{5}} 
\subsubsection*{\textit{Решение:}}

Суммарная длина простых делителей $p-1$ не превышает $\log{p}$. Самих уровней в дереве - не более $\log{p}$: минимальный простой делитель 2. Для нахождения порождающего элемента быстро возводим в степень за $O(\log{p})$ операций умножения и $O(\log{p})$ операций взятия остатка, сложность каждой операции $O(\log^2{p})$, поэтому за $O(\log^3{p})$ можно подсчитать $k^{p-1}$, где $k$ - порождающий элемент. Всего чисел $k^{\frac{p-1}{p_i}}$(, где $p_i$ - простые делители числа $p-1$) $O(\log{p})$, значит проверить истинность всех можно за $O(\log^4{p})$. Проверка того, что простые делители $p-1$ есть $p_i$ и только они делается за $O(\log^3{p})$. Поэтому:

\[T(p) = O(\log{p}) + \sum_{i=1}^k T(p_i)\]

Т.к. $x_1^k+\dots+x_n^k \leq (x_1 + \dots + x_n)^k$, то сумму на каждом уровне оценим в $O(\log^4{p})$, всего уровней $\log{p}$. Получаем, что сложность проверки сертификата $O(\log^5{p}) \Rightarrow$ полиномиальна.

\[100091237, 100091236 = 2^2 * 7 * 3574687, k = 2\]

\[7, 6 = 2 * 3, k = 2\]

\[3574687, 3574686 = 2 * 3 * 233 * 2557, k = 62\]

\[233, 232 = 2^3 * 29, k =2\]

\[29, 28 = 2^2 * 7, k = 2\]

\[2557, 2556 = 2^2 * 3^2 * 71, k = 2\]

\[71, 70 = 2 * 5 * 7, k = 14\]

\subsection*{\circled{6}} 
\subsubsection*{\textit{Решение:}}

Ссылки на решение задач $\href{https://bit.ly/2SBvEmS}{One}, \href{https://bit.ly/2Xwzy45}{Two}$

\subsection*{\circled{7}} 
\subsubsection*{\textit{Решение:}}

Язык $L$ лежит в классе $NP$, следовательно, для него существует функция $V(x, s)$ с булевыми значениями, вычислимая за полиномиальное время от длины первого аргумента, такая, что:
	 \[x \in L \to \exists s: V(x, s) = 1;\]
	\[x \notin L \to \forall s: V(x, s) = 0;\]

\begin{enumerate}
\item На вход алгоритму приходит описания 2-ух графов. Требуется проверить их на изоморфность. Следовательно, нужно предоставить сертификат полиномиального размера от длины входа.

Сертификат - две строки массивов длины $|V|$, верхняя строка - вершины в произвольном (заданном) порядке, вторая - куда они отображаются. Далее, пройдя по графу, проверяем наличие путей в изначальном графе и в изменном по функции-сертификату.

\item Сертификат - размер клики $k$. Далее проходом по графу мы проверяем каждую вершину, есть ли от нее $k$ ребер. Записываем им в определенный массив и на следующем шаге (переходом по одному из ребер) проверяем, будет ли от следующей вершины тоже $k$ ребер. Если из вершины нет $k$ ребер, то удаляем ее из списка вершин. Таким образом мы сведем граф к полному.

\item Приведем систему к ступенчатому виду. Далее, если система имеет экзотические уравнения ($\href{https://www.intuit.ru/studies/courses/1009/197/lecture/5126}{Click}$). Это мы сделаем за полином и по критерию найдем ответ.
\end{enumerate}
\end{document}
